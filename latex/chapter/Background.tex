\chapter{Grundlagen}
\label{ch:basics}

\newcommand{\mean}{\bar{x}_{arithm}}

% \section{Fähigkeiten von Llama}

% \begin{itemize}
%     \item ChatDoctor: A Medical Chat Model Fine-Tuned on a Large Language Model Meta-AI (LLaMA) Using Medical Domain Knowledge \citep{liChatDoctorMedicalChat2023}
%     \item Code Llama: Open Foundation Models for Code \citep{roziereCodeLlamaOpen2024}
%     \item The Llama 3 Herd of Models | Research - AI at Meta \citep{ai@metaLlama3Herd2024}
% \end{itemize}

% \section{Geografische Fähigkeiten von Large Language Models}

% \begin{itemize}
%     \item Are Large Language Models Geospatially Knowledgeable? \citep{bhandariAreLargeLanguage2023}
%     \item Evaluating Spatial Understanding of Large Language Models \citep{yamadaEvaluatingSpatialUnderstanding2024}
%     \item GeoLLM: Extracting Geospatial Knowledge from Large Language Models \citep{manviGeoLLMExtractingGeospatial2024}
%     \item GPT4GEO: How a Language Model Sees the World's Geography \citep{robertsGPT4GEOHowLanguage2023}
% \end{itemize}

In diesem Kapitel werden einige grundlegende Konzepte vorgestellt, auf denen diese Arbeit aufbaut.
Zunächst werden die wesentlichen Grundlagen zu LLMs erläutert, gefolgt von einem Überblick über das geographische Basiswissen.
Abschließend werden die in dieser Arbeit genutzten Evaluationsmetriken sowie einige technische Grundlagen präsentiert.

\section*{Grundlagen zu Large Language Models (LLMs)}
% Architektur (z. B. Transformer, Attention-Mechanismen)
Mit \textit{Large Language Models (LLMs)} sind meistens \textit{Language Models} basierend auf Transformern gemeint, die Hunderte Milliarden (oder mehr) Parameter enthalten und mithilfe von sehr vielen Textdaten trainiert werden.
LLMs verfügen über ausgeprägte Fähigkeiten, natürliche Sprache zu verstehen und komplexe Aufgaben (durch Textgenerierung) zu lösen \citep{zhaoSurveyLargeLanguage2024}.
Weitere Informationen über die Entwicklung, Fähigkeiten und wichtige Techniken in Bezug auf LLMs liefert die Arbeit \textit{A Survey of Large Language Models} von \citet{zhaoSurveyLargeLanguage2024}.

% Typically, large language models (LLMs) refer to Transformer language models that contain hundreds of billions (or more) of parameters4, which are trained on massive text data [32], such as GPT-3 [55], PaLM [56], Galactica [35], and LLaMA [57]. LLMs exhibit strong capacities to un- derstand natural language and solve complex tasks (via text generation).

% LLaMA (Versionen, Parametergrößen)
Meta hat mittlerweile drei Versionen an LLMs veröffentlicht: LLaMA \citep{touvronLLaMAOpenEfficient2023}, Llama 2 \citep{touvronLlama2Open2023} und Llama 3 \citep{ai@metaLlama3Herd2024}. Details zu den einzelnen Versionen und Modellen können den zugehörigen Quellen entnommen werden.

Von den Llama 3 Modellen gibt es verschiedene Ausführungen: Llama 3, Llama 3.1, Llama 3.2 und Llama 3.3\footnote{Details: \url{https://github.com/meta-llama/llama-models/blob/main/models/}}.
Von diesen Modellarten bieten die Llama 3.1 Modelle mit den Parametergrößen 8~Mrd., 70~Mrd. und 405~Mrd. das größte Spektrum.
Das Llama 3.3 Modell gibt es nur mit 70~Mrd. Parametern und die Llama 3.2 Modelle mit 1~Mrd. und 3~Mrd. Parametern.
Außerdem existieren Llama 3.3 Modelle, die auch Bilder als Eingabe verwenden können.

% Anfragen an LLMs (inference, system prompt, Nutzereingabe)
Anfragen an LLMs erfolgen meist in Form von strukturierten Texteingaben.
Für die Llama 3 Modelle gibt es beispielsweise vier vordefinierte Rollen für die Textabschnitte: \textbf{system} setzt den Kontext (z.\,B. Regeln und allgemeine Informationen), \textbf{user} ist die Eingabe des Nutzers, \textbf{ipython} wird bei der Nutzung externer Werkzeuge eingesetzt und \textbf{assistant} enthält die Antwort des Modells.
Weitere Informationen zur Nutzung der Llama Modelle sind unter \url{https://www.llama.com/docs/} verfügbar.

Zudem existieren auch alternative APIs, wie unter anderem die OpenAI API\footnote{\url{https://platform.openai.com/docs/overview}}, die beispielsweise von DeepInfra\footnote{\url{https://deepinfra.com/}} genutzt wird.
DeepInfra ermöglicht durch diese Standardisierung eine einfache Interaktion mit einer Vielzahl verschiedener LLMs.



\section*{Geographisches Basiswissen}
Da diese Arbeit sich mit den geographischen Fähigkeiten von LLMs beschäftigt, ist ein grundlegendes Verständnis von einigen geographischen Konzepten wichtig.
Im Folgenden werden die geographischen Koordinaten, Ländercodes und Berechnungsverfahren für geographische Distanzen vorgestellt.

% geographische Koordinaten
Die geographischen Koordinaten, die in dieser Arbeit genutzt werden, sind Kugelkoordinaten, die eine Position auf der Erdoberfläche beschreiben.
Die Koordinaten bestehen aus Längen- und Breitengrad.
Weitere Informationen zu geographischen Koordinaten, wie z.\,B. zu verschiedenen Darstellungsmöglichkeiten und was genau Längen- und Breitengrad bedeuten, sind unter \url{https://de.wikipedia.org/wiki/Geographische_Koordinaten} zu finden.

% Berechnungsverfahren für geographische Distanzen
Um die Distanz zwischen zwei geographischen Koordinaten zu berechnen, gibt es unter Anderem die Möglichkeit, die kürzeste Distanz dieser Punkte auf einer Kugeloberfläche zu bestimmen (auch Orthodrom genannt\footnote{\url{https://de.wikipedia.org/wiki/Orthodrome}}).
Da die Erde keine perfekte Kugel ist, werden genauere Ergebnisse erzielt, wenn man statt einer Kugel das WGS84-Ellipsoid zugrunde legt.
Zur Berechnung kann dabei der Algorithmus von \citet{karneyAlgorithmsGeodesics2013} verwendet werden.

% Ländercodes
Manchmal wird anstatt eines Ländernamens der zugehörige Ländercode nach der ISO-3166-1-Kodierliste (Ländercodeliste)\footnote{\url{https://de.wikipedia.org/wiki/ISO-3166-1-Kodierliste}} verwendet.
Die Standardisierung hat den Vorteil, dass Probleme mit verschiedenen Bezeichnungen für ein Land in verschiedenen Sprachen sowie allgemeine Eindeutigkeitsprobleme behoben werden.



\section*{Evaluationsmetriken}
Um die Genauigkeit und Streuung der Antworten zu beschreiben, werden in dieser Arbeit zwei grundlegende statistische Maße verwendet: das \textbf{arithmetische Mittel} (Durchschnitt) und die \textbf{Standardabweichung}.

Das arithmetische Mittel\footnote{\url{https://de.wikipedia.org/wiki/Mittelwert}} (auch Mittelwert oder Durchschnitt genannt) ist die Summe der gegebenen Werte geteilt durch die Anzahl der Werte und gibt damit an, wie hoch der typische Wert eines Datensatzes ist.
\[ \mean = \frac{1}{n} \sum_{i = 1}^{n} x_i = \frac{x_1 + x_2 + \cdots + x_n}{n} \]
In den nachfolgenden Experimenten wird in der Regel das arithmetische Mittel von Daten ermittelt, die an manchen Stellen keine Zahlen enthalten, da zuvor Fehler aufgetreten sind.
Diese Werte werden bei der Berechnung ignoriert\footnote{Mithilfe von: \url{https://numpy.org/doc/2.0/reference/generated/numpy.nanmean.html}}.

Die Standardabweichung\footnote{\url{https://de.wikipedia.org/wiki/Varianz_(Stochastik)}} ist die Quadratwurzel der Varianz und eins der wichtigsten Streuungsmaße der Stochastik.
Sie beschreibt, wie stark die einzelnen Datenpunkte um das arithmetische Mittel streuen.
\[ \sigma = \sqrt{ \frac{1}{n} \sum_{i = 1}^{n} \left( x_i - \mean \right)^2 } \]
Auch bei der Berechnung der Standardabweichung werden häufig Daten mit Lücken betrachtet, die dabei ignoriert werden\footnote{Mithilfe von: \url{https://numpy.org/doc/2.1/reference/generated/numpy.nanstd.html}}.



\section*{Technische Grundlagen}
In dieser Arbeit werden das JSON-Format sowie reguläre Ausdrücke zur Datenextraktion aus Texten verwendet.

Das JSON-Format wird hauptsächlich zur Speicherung und Übertragung strukturierter Daten eingesetzt.
Ein einfaches Beispiel aus dem Kapitel \ref{ch:results} ist in der Abbildung \ref{dist_answer_background} dargestellt.
Dabei ist \textit{""distance""} der Schlüssel, mit dem der Wert \textit{6171.479892} abgerufen werden kann.
Auf diese Weise können leicht einzelne Werte aus strukturierten Daten extrahiert werden.
Weitere Informationen zum JSON-Format findet man z.\,B. unter \url{https://de.wikipedia.org/wiki/JSON}.

\begin{figure} % Antwort für distance

    \begin{lstlisting}[literate={°}{\textdegree}1, breaklines=true]
        {
            ""distance"": 6171.479892
        }
    \end{lstlisting}

    \caption{Die Antwort des Llama-3.1-8B-Instruct Modells für die Anfrage \ref{dist_message}.}

    \label{dist_answer_background}
\end{figure}

Reguläre Ausdrücke (Abkürzung: Regex) sind eine Möglichkeit, ein festes Muster zu beschreiben, mit dem unter anderem in einem Text gesucht werden kann.
Zum Beispiel stellt der reguläre Ausdruck \textbf{(ab)*} alle Wörter dar, die aus beliebig vielen Wiederholungen von \textbf{ab} bestehen.
Mit deutlich komplexeren regulären Ausdrücken kann man verschiedene Koordinatenformate in unstrukturierten Texten suchen und die Koordinaten anschließend extrahieren.
Weitere Informationen über die Semantik und den Nutzen von regulären Ausdrücken sind unter \url{https://de.wikipedia.org/wiki/Regul%C3%A4rer_Ausdruck} zu finden.