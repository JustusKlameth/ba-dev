\begin{abstract}
\section*{Zusammenfassung}

% \section*{Tips}

% The abstract is an overview and summary of your thesis. It should be kept short and should highlight the main contents as well as findings. For citations see for example  \citep{jele2010}.

Diese Arbeit untersucht die geographischen Fähigkeiten ausgewählter Llama 3 Modelle mit dem Ziel, deren Genauigkeit bei der Bestimmung von Stadtkoordinaten und Distanzen zwischen Städten zu analysieren.
Dazu wurden mithilfe eines geeigneten Datensatzes verschiedene Anfrageformate und Auswertungsverfahren genutzt, um Antworten der Modelle zu sammeln.
Diese wurden systematisch ausgewertet, um die Leistungsfähigkeit der Modelle zu ermitteln.

Die Ergebnisse zeigen, dass für die Stadtkoordinaten mit ausgewählten Modellen, Anfrage- und Auswertungsverfahren ein durchschnittlicher Fehler von unter 100 km erreicht werden kann, was im globalen Kontext als akzeptabel betrachtet werden könnte.
Bei den Distanzen zwischen Städten liegen die durchschnittlichen Fehler bei über 600~km, was auch im globalen Maßstab als zu ungenau einzustufen ist.
Besonders auffällig ist die hohe Standardabweichung in allen durchgeführten Experimenten.
Dadurch sind die getesteten Verfahren für den praktischen Einsatz noch ungeeignet, da nicht nur ein möglichst geringer durchschnittlicher Fehler, sondern auch akkurate Einzelvorhersagen erforderlich sind.

% Zukünftige Arbeiten könnten untersuchen, wie sich die Antworten bei einem allgemeineren Datensatz verhalten, und alternative Methoden zur Reduzierung der Standardabweichung entwickeln, um die Modelle für praktische Anwendungen nutzbar zu machen.

Die Arbeit liefert damit einen Einblick in die geographischen Fähigkeiten ausgewählter Llama 3 Modelle und bildet eine Grundlage für zukünftige Untersuchungen und Vergleiche.

\end{abstract}
