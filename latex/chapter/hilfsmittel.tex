%!TEX root = ../template.tex
\chapter*{Hilfsmittel}

Für diese Arbeit wurden verschiedene Hilfsmittel eingesetzt, um sowohl sprachliche als auch strukturelle Aspekte zu optimieren.
Hier wird ein kurzer Überblick über die wichtigsten verwendeten Hilfsmittel gegeben:

\begin{enumerate}
    
    \item \textbf{DeepL}\footnote{\url{https://www.deepl.com/de/translator}} wurde hauptsächlich bei der Literaturrecherche zur Übersetzung vom Englischen ins Deutsche verwendet und zur Suche nach deutschen Formulierungen für englische Ausdrücke.

    \item \textbf{DeepL Write}\footnote{\url{https://www.deepl.com/de/write}} wurde als Formulierungshilfe genutzt.
    Dabei war besonders die Funktion \textit{Wort austauschen} hilfreich, um passende Synonyme bei Wortwiederholungen zu finden.

    \item \textbf{ChatGPT}\footnote{\url{https://chatgpt.com/}} wurde als nützliches Hilfsmittel zum Brainstormen verwendet.
    Dabei wurde zum Beispiel über viele Versionen des Titels und der Fragestellung dieser Arbeit iteriert.
    Außerdem wurden im Rahmen eines Dialoges mögliche Kapitelaufteilungen erstellt, die tatsächliche Ausformulierung ist aber immer selbständig erfolgt.

\end{enumerate}